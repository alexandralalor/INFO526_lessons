% Options for packages loaded elsewhere
\PassOptionsToPackage{unicode}{hyperref}
\PassOptionsToPackage{hyphens}{url}
%
\documentclass[
]{article}
\usepackage{amsmath,amssymb}
\usepackage{lmodern}
\usepackage{iftex}
\ifPDFTeX
  \usepackage[T1]{fontenc}
  \usepackage[utf8]{inputenc}
  \usepackage{textcomp} % provide euro and other symbols
\else % if luatex or xetex
  \usepackage{unicode-math}
  \defaultfontfeatures{Scale=MatchLowercase}
  \defaultfontfeatures[\rmfamily]{Ligatures=TeX,Scale=1}
\fi
% Use upquote if available, for straight quotes in verbatim environments
\IfFileExists{upquote.sty}{\usepackage{upquote}}{}
\IfFileExists{microtype.sty}{% use microtype if available
  \usepackage[]{microtype}
  \UseMicrotypeSet[protrusion]{basicmath} % disable protrusion for tt fonts
}{}
\makeatletter
\@ifundefined{KOMAClassName}{% if non-KOMA class
  \IfFileExists{parskip.sty}{%
    \usepackage{parskip}
  }{% else
    \setlength{\parindent}{0pt}
    \setlength{\parskip}{6pt plus 2pt minus 1pt}}
}{% if KOMA class
  \KOMAoptions{parskip=half}}
\makeatother
\usepackage{xcolor}
\IfFileExists{xurl.sty}{\usepackage{xurl}}{} % add URL line breaks if available
\IfFileExists{bookmark.sty}{\usepackage{bookmark}}{\usepackage{hyperref}}
\hypersetup{
  pdftitle={Scatterplots and Barplots},
  pdfauthor={Adriana Picoral},
  hidelinks,
  pdfcreator={LaTeX via pandoc}}
\urlstyle{same} % disable monospaced font for URLs
\usepackage[margin=1in]{geometry}
\usepackage{color}
\usepackage{fancyvrb}
\newcommand{\VerbBar}{|}
\newcommand{\VERB}{\Verb[commandchars=\\\{\}]}
\DefineVerbatimEnvironment{Highlighting}{Verbatim}{commandchars=\\\{\}}
% Add ',fontsize=\small' for more characters per line
\usepackage{framed}
\definecolor{shadecolor}{RGB}{248,248,248}
\newenvironment{Shaded}{\begin{snugshade}}{\end{snugshade}}
\newcommand{\AlertTok}[1]{\textcolor[rgb]{0.94,0.16,0.16}{#1}}
\newcommand{\AnnotationTok}[1]{\textcolor[rgb]{0.56,0.35,0.01}{\textbf{\textit{#1}}}}
\newcommand{\AttributeTok}[1]{\textcolor[rgb]{0.77,0.63,0.00}{#1}}
\newcommand{\BaseNTok}[1]{\textcolor[rgb]{0.00,0.00,0.81}{#1}}
\newcommand{\BuiltInTok}[1]{#1}
\newcommand{\CharTok}[1]{\textcolor[rgb]{0.31,0.60,0.02}{#1}}
\newcommand{\CommentTok}[1]{\textcolor[rgb]{0.56,0.35,0.01}{\textit{#1}}}
\newcommand{\CommentVarTok}[1]{\textcolor[rgb]{0.56,0.35,0.01}{\textbf{\textit{#1}}}}
\newcommand{\ConstantTok}[1]{\textcolor[rgb]{0.00,0.00,0.00}{#1}}
\newcommand{\ControlFlowTok}[1]{\textcolor[rgb]{0.13,0.29,0.53}{\textbf{#1}}}
\newcommand{\DataTypeTok}[1]{\textcolor[rgb]{0.13,0.29,0.53}{#1}}
\newcommand{\DecValTok}[1]{\textcolor[rgb]{0.00,0.00,0.81}{#1}}
\newcommand{\DocumentationTok}[1]{\textcolor[rgb]{0.56,0.35,0.01}{\textbf{\textit{#1}}}}
\newcommand{\ErrorTok}[1]{\textcolor[rgb]{0.64,0.00,0.00}{\textbf{#1}}}
\newcommand{\ExtensionTok}[1]{#1}
\newcommand{\FloatTok}[1]{\textcolor[rgb]{0.00,0.00,0.81}{#1}}
\newcommand{\FunctionTok}[1]{\textcolor[rgb]{0.00,0.00,0.00}{#1}}
\newcommand{\ImportTok}[1]{#1}
\newcommand{\InformationTok}[1]{\textcolor[rgb]{0.56,0.35,0.01}{\textbf{\textit{#1}}}}
\newcommand{\KeywordTok}[1]{\textcolor[rgb]{0.13,0.29,0.53}{\textbf{#1}}}
\newcommand{\NormalTok}[1]{#1}
\newcommand{\OperatorTok}[1]{\textcolor[rgb]{0.81,0.36,0.00}{\textbf{#1}}}
\newcommand{\OtherTok}[1]{\textcolor[rgb]{0.56,0.35,0.01}{#1}}
\newcommand{\PreprocessorTok}[1]{\textcolor[rgb]{0.56,0.35,0.01}{\textit{#1}}}
\newcommand{\RegionMarkerTok}[1]{#1}
\newcommand{\SpecialCharTok}[1]{\textcolor[rgb]{0.00,0.00,0.00}{#1}}
\newcommand{\SpecialStringTok}[1]{\textcolor[rgb]{0.31,0.60,0.02}{#1}}
\newcommand{\StringTok}[1]{\textcolor[rgb]{0.31,0.60,0.02}{#1}}
\newcommand{\VariableTok}[1]{\textcolor[rgb]{0.00,0.00,0.00}{#1}}
\newcommand{\VerbatimStringTok}[1]{\textcolor[rgb]{0.31,0.60,0.02}{#1}}
\newcommand{\WarningTok}[1]{\textcolor[rgb]{0.56,0.35,0.01}{\textbf{\textit{#1}}}}
\usepackage{graphicx}
\makeatletter
\def\maxwidth{\ifdim\Gin@nat@width>\linewidth\linewidth\else\Gin@nat@width\fi}
\def\maxheight{\ifdim\Gin@nat@height>\textheight\textheight\else\Gin@nat@height\fi}
\makeatother
% Scale images if necessary, so that they will not overflow the page
% margins by default, and it is still possible to overwrite the defaults
% using explicit options in \includegraphics[width, height, ...]{}
\setkeys{Gin}{width=\maxwidth,height=\maxheight,keepaspectratio}
% Set default figure placement to htbp
\makeatletter
\def\fps@figure{htbp}
\makeatother
\setlength{\emergencystretch}{3em} % prevent overfull lines
\providecommand{\tightlist}{%
  \setlength{\itemsep}{0pt}\setlength{\parskip}{0pt}}
\setcounter{secnumdepth}{-\maxdimen} % remove section numbering
\ifLuaTeX
  \usepackage{selnolig}  % disable illegal ligatures
\fi

\title{Scatterplots and Barplots}
\usepackage{etoolbox}
\makeatletter
\providecommand{\subtitle}[1]{% add subtitle to \maketitle
  \apptocmd{\@title}{\par {\large #1 \par}}{}{}
}
\makeatother
\subtitle{INFO 526 Data Analysis and Visualization}
\author{Adriana Picoral}
\date{}

\begin{document}
\maketitle

In this tutorial, we will focus on two types of visualizations: bar
plots, and scatterplots. Like all the other tutorials in this course,
this document is accompany by code-along videos, and the source code (to
be found on D2L).

The learning outcomes of this tutorial include:

\begin{itemize}
\tightlist
\item
  Produce effective bar charts and scatterplots
\item
  Identify visualization errors and pitfalls
\end{itemize}

\hypertarget{scatterplots}{%
\section{Scatterplots}\label{scatterplots}}

For our first scatterplot, we will be using the \texttt{openintro}
library which accompanies Çetinkaya-Rundel and Hardin's (2021) textbook
(first chapter can be found to download on D2L, full book can be
accessed at \url{https://openintro-ims.netlify.app/}). We will also load
\texttt{scales} for some plot formatting, and \texttt{tidyverse} for the
\texttt{ggplot} function.

\begin{Shaded}
\begin{Highlighting}[]
\FunctionTok{library}\NormalTok{(openintro)}
\FunctionTok{library}\NormalTok{(scales)}
\FunctionTok{library}\NormalTok{(tidyverse)}
\end{Highlighting}
\end{Shaded}

Many packages come with data in addition to functions. The package
\texttt{openintro} contains all datasets used in
\href{https://openintro-ims.netlify.app/}{Introduction to Modern
Statistics}. We will be working with the \texttt{county}data, so let's
load that using the \texttt{data()} function.

\begin{Shaded}
\begin{Highlighting}[]
\CommentTok{\# load data from openintro packages}
\FunctionTok{data}\NormalTok{(}\StringTok{"county"}\NormalTok{)}
\end{Highlighting}
\end{Shaded}

We will now recreate Figure 1.2 in
\href{https://openintro-ims.netlify.app/}{Introduction to Modern
Statistics}. You can see the actual code that was used to render this
figure in the
\href{https://github.com/OpenIntroStat/ims/blob/main/01-data-hello.Rmd}{textbook's
GitHub repository}. For color names, I use
\href{http://www.stat.columbia.edu/~tzheng/files/Rcolor.pdf}{this
guide}.

For scatterplots, you usually plot two continuous numeric variables. In
this case we are plotting 1) the percent of housing units that are
multi-unit structures in different counties in the United States between
2006 and 2010, and 2) homeownership percentage for the same counties and
time frame. We will use \texttt{geom\_point} to represent each data
point with a point for the x and y values. We use \texttt{anotate} and
\texttt{geom\_text} to replicate the original annotation in red (to
highlight a specific data point). Finally we do some plot formatting
using \texttt{theme\_}, \texttt{labs}, and \texttt{scale\_}.

\begin{Shaded}
\begin{Highlighting}[]
\CommentTok{\# recreate Figure 1.2}
\NormalTok{county }\SpecialCharTok{\%\textgreater{}\%}
  \FunctionTok{ggplot}\NormalTok{(}\FunctionTok{aes}\NormalTok{(}\AttributeTok{x =}\NormalTok{ multi\_unit,}
             \AttributeTok{y =}\NormalTok{ homeownership)) }\SpecialCharTok{+}
  \FunctionTok{geom\_point}\NormalTok{(}\AttributeTok{color =} \StringTok{"royalblue4"}\NormalTok{,}
             \AttributeTok{fill =} \StringTok{"dodgerblue3"}\NormalTok{,}
             \AttributeTok{shape =} \DecValTok{21}\NormalTok{,}
             \AttributeTok{alpha =}\NormalTok{ .}\DecValTok{3}\NormalTok{) }\SpecialCharTok{+}
  \FunctionTok{annotate}\NormalTok{(}\StringTok{"segment"}\NormalTok{,}
           \AttributeTok{x =} \FloatTok{39.4}\NormalTok{, }\AttributeTok{xend =} \FloatTok{39.4}\NormalTok{,}
           \AttributeTok{y =} \DecValTok{0}\NormalTok{, }\AttributeTok{yend =} \FloatTok{31.3}\NormalTok{,}
           \AttributeTok{color =} \StringTok{"red"}\NormalTok{,}
           \AttributeTok{linetype =} \StringTok{"dashed"}\NormalTok{) }\SpecialCharTok{+}
  \FunctionTok{annotate}\NormalTok{(}\StringTok{"segment"}\NormalTok{,}
           \AttributeTok{x =} \DecValTok{0}\NormalTok{, }\AttributeTok{xend =} \FloatTok{39.4}\NormalTok{,}
           \AttributeTok{y =} \FloatTok{31.3}\NormalTok{, }\AttributeTok{yend =} \FloatTok{31.3}\NormalTok{,}
           \AttributeTok{color =} \StringTok{"red"}\NormalTok{,}
           \AttributeTok{linetype =} \StringTok{"dashed"}\NormalTok{) }\SpecialCharTok{+}
  \FunctionTok{geom\_point}\NormalTok{(}\AttributeTok{x =} \FloatTok{39.4}\NormalTok{, }\AttributeTok{y =} \FloatTok{31.3}\NormalTok{,}
             \AttributeTok{shape =} \DecValTok{21}\NormalTok{, }\AttributeTok{color =} \StringTok{"red"}\NormalTok{) }\SpecialCharTok{+}
  \FunctionTok{geom\_text}\NormalTok{(}\AttributeTok{label =} \StringTok{"Chattahoochee County"}\NormalTok{, }\AttributeTok{fontface =} \StringTok{"italic"}\NormalTok{,}
            \AttributeTok{x =} \DecValTok{55}\NormalTok{, }\AttributeTok{y =} \DecValTok{30}\NormalTok{, }\AttributeTok{color =} \StringTok{"red"}\NormalTok{) }\SpecialCharTok{+}
  \FunctionTok{theme\_minimal}\NormalTok{() }\SpecialCharTok{+}
  \FunctionTok{labs}\NormalTok{(}\AttributeTok{x =} \StringTok{"Percent of housing units that are muti{-}unit structures"}\NormalTok{,}
        \AttributeTok{y =} \StringTok{"Homeownership rate"}\NormalTok{) }\SpecialCharTok{+}
  \FunctionTok{scale\_x\_continuous}\NormalTok{(}\AttributeTok{labels =} \FunctionTok{percent\_format}\NormalTok{(}\AttributeTok{scale =} \DecValTok{1}\NormalTok{)) }\SpecialCharTok{+}
  \FunctionTok{scale\_y\_continuous}\NormalTok{(}\AttributeTok{labels =} \FunctionTok{percent\_format}\NormalTok{(}\AttributeTok{scale =} \DecValTok{1}\NormalTok{)) }
\end{Highlighting}
\end{Shaded}

\begin{figure}

{\centering \includegraphics{scatterplots-barplots_code-along_files/figure-latex/ims_fig-1} 

}

\caption{A scatterplot of homeownership versus the percent of housing units that are in multi-unit structures for US counties.}\label{fig:ims_fig}
\end{figure}

\hypertarget{association-does-not-equal-causation}{%
\subsection{Association does not equal
causation}\label{association-does-not-equal-causation}}

Scatterplots are meant to display the association between two variables.
What we have to keep in mind, however, is that association (or
correlation) is not the same as causation. There is a lot about
\href{http://tylervigen.com/spurious-correlations}{Spurious
correlations}. In this tutorial we will work with the classic
\emph{number of people who drowned correlates with number of filmes
Nicolas cage appeared in overtime}. Here's the two data sets I was able
to find:

\href{https://www.kaggle.com/eharlett/nic-cage-movies}{Nicolas Cage's
movies}

\href{https://www.cdc.gov/nchs/data/databriefs/db149_table.pdf\#1}{Unintentional
Drowning Deaths in the United States, 1999--2010}

\begin{Shaded}
\begin{Highlighting}[]
\CommentTok{\# read data in}
\NormalTok{drowning\_deaths\_nic\_cage\_movies }\OtherTok{\textless{}{-}} \FunctionTok{read\_csv}\NormalTok{(}\StringTok{"data\_raw/drowning\_deaths\_nic\_cage\_movies.csv"}\NormalTok{)}
\end{Highlighting}
\end{Shaded}

The problem here is that I was unable to find the original data sets
(problem of transparency). Anyway, let's calculate the correlation of
these two variables with the data we have.

\begin{Shaded}
\begin{Highlighting}[]
\NormalTok{drowning\_deaths\_nic\_cage\_movies }\SpecialCharTok{\%\textgreater{}\%}
  \FunctionTok{summarize}\NormalTok{(}\AttributeTok{correlation =} \FunctionTok{cor}\NormalTok{(cage\_movie\_count, deaths))}
\end{Highlighting}
\end{Shaded}

\begin{verbatim}
## # A tibble: 1 x 1
##   correlation
##         <dbl>
## 1       0.113
\end{verbatim}

\begin{Shaded}
\begin{Highlighting}[]
\FunctionTok{cor.test}\NormalTok{(drowning\_deaths\_nic\_cage\_movies}\SpecialCharTok{$}\NormalTok{cage\_movie\_count, }
\NormalTok{    drowning\_deaths\_nic\_cage\_movies}\SpecialCharTok{$}\NormalTok{deaths)}
\end{Highlighting}
\end{Shaded}

\begin{verbatim}
## 
##  Pearson's product-moment correlation
## 
## data:  drowning_deaths_nic_cage_movies$cage_movie_count and drowning_deaths_nic_cage_movies$deaths
## t = 0.36023, df = 10, p-value = 0.7262
## alternative hypothesis: true correlation is not equal to 0
## 95 percent confidence interval:
##  -0.4927230  0.6451773
## sample estimates:
##       cor 
## 0.1131842
\end{verbatim}

We will now try to recreate the original plot.

\begin{Shaded}
\begin{Highlighting}[]
\CommentTok{\# plot it}
\NormalTok{drowning\_deaths\_nic\_cage\_movies }\SpecialCharTok{\%\textgreater{}\%}
  \FunctionTok{ggplot}\NormalTok{(}\FunctionTok{aes}\NormalTok{(}\AttributeTok{x =}\NormalTok{ year)) }\SpecialCharTok{+}
  \FunctionTok{geom\_point}\NormalTok{(}\FunctionTok{aes}\NormalTok{(}\AttributeTok{y =}\NormalTok{ cage\_movie\_count)) }\SpecialCharTok{+}
  \FunctionTok{geom\_point}\NormalTok{(}\FunctionTok{aes}\NormalTok{(}\AttributeTok{y =}\NormalTok{ deaths}\SpecialCharTok{/}\DecValTok{1600}\NormalTok{),}
             \AttributeTok{color =} \StringTok{"blue"}\NormalTok{) }\SpecialCharTok{+}
  \FunctionTok{scale\_y\_continuous}\NormalTok{(}\AttributeTok{name =} \StringTok{"Nicholas Cage movie count"}\NormalTok{,}
                     \AttributeTok{sec.axis =} \FunctionTok{sec\_axis}\NormalTok{(}\SpecialCharTok{\textasciitilde{}}\NormalTok{.}\SpecialCharTok{*}\DecValTok{1600}\NormalTok{, }\AttributeTok{name=}\StringTok{"deaths by drowning in the US"}\NormalTok{),}
                     \AttributeTok{limits =} \FunctionTok{c}\NormalTok{(}\DecValTok{0}\NormalTok{, }\DecValTok{6}\NormalTok{)) }\SpecialCharTok{+}
  \FunctionTok{scale\_x\_continuous}\NormalTok{(}\AttributeTok{breaks =} \FunctionTok{c}\NormalTok{(}\DecValTok{1999}\SpecialCharTok{:}\DecValTok{2010}\NormalTok{)) }\SpecialCharTok{+}
  \FunctionTok{theme\_linedraw}\NormalTok{()}
\end{Highlighting}
\end{Shaded}

\begin{figure}
\centering
\includegraphics{scatterplots-barplots_code-along_files/figure-latex/nic_cage_fig-1.pdf}
\caption{Number of Nicholas Cage movies and number of deaths by drowning
in the US.}
\end{figure}

Not great. The challenge stands: if you are able to find the specific
data set on drownings that they use, and recreate the original plot, let
everyone one in the discussion board.

\hypertarget{bar-plots}{%
\section{Bar plots}\label{bar-plots}}

Barplots are often used to visualize the frequency distribution of a
categorical variable. The idea is that we count different categories
(i.e., levels) within a categorical variable.

There are geometrics functions within \texttt{ggplot} that allow you to
draw up bar plots:

\begin{itemize}
\tightlist
\item
  \texttt{geom\_bar()} counts frequencies by levels within a categorical
  variable and plot those frequencies. You need to map the categorical
  variable to one of the axes only (the other axis is unmapped, since it
  will take the frequency values automatically)
\item
  \texttt{geom\_col()} requires the two axes to be mapped, one axis is
  mapped to the categoric variable the other is mapped to the count
  variable.
\end{itemize}

\hypertarget{using-geom_bar}{%
\subsection{Using geom\_bar()}\label{using-geom_bar}}

For this short demonstration, we will be using the \texttt{penguins}
dataset again. Let's load \texttt{tidyverse} and
\texttt{palmerpenguins}.

\begin{Shaded}
\begin{Highlighting}[]
\FunctionTok{library}\NormalTok{(palmerpenguins)}
\end{Highlighting}
\end{Shaded}

As mentioned, you need to map one of the axes to the categorical
variable for which you want to visualize the frequency distribution.
Let's visualize the frequency of each species in the \texttt{penguins}
dataset, mapping \texttt{species} to \texttt{x} first and then adding
\texttt{geom\_col()} to the code block.

\begin{Shaded}
\begin{Highlighting}[]
\NormalTok{penguins }\SpecialCharTok{\%\textgreater{}\%}
  \FunctionTok{ggplot}\NormalTok{(}\FunctionTok{aes}\NormalTok{(}\AttributeTok{x =}\NormalTok{ species)) }\SpecialCharTok{+}
  \FunctionTok{geom\_bar}\NormalTok{()}
\end{Highlighting}
\end{Shaded}

\begin{figure}

{\centering \includegraphics{scatterplots-barplots_code-along_files/figure-latex/geom_bar_plot-1} 

}

\caption{Straight-forward plot of penguin specie count.}\label{fig:geom_bar_plot}
\end{figure}

Note that we can map the categorical variable to \texttt{y} instead.

\begin{Shaded}
\begin{Highlighting}[]
\NormalTok{penguins }\SpecialCharTok{\%\textgreater{}\%}
  \FunctionTok{ggplot}\NormalTok{(}\FunctionTok{aes}\NormalTok{(}\AttributeTok{y =}\NormalTok{ species)) }\SpecialCharTok{+}
  \FunctionTok{geom\_bar}\NormalTok{()}
\end{Highlighting}
\end{Shaded}

\begin{figure}

{\centering \includegraphics{scatterplots-barplots_code-along_files/figure-latex/horizontal_bar_plot-1} 

}

\caption{Horizontal bar plot of penguin specie count.}\label{fig:horizontal_bar_plot}
\end{figure}

Note that the counts calculated in \texttt{geom\_col()} are the same as
if we were to calculate the counts ourselves.

\begin{Shaded}
\begin{Highlighting}[]
\NormalTok{penguins }\SpecialCharTok{\%\textgreater{}\%}
  \FunctionTok{count}\NormalTok{(species)}
\end{Highlighting}
\end{Shaded}

\begin{verbatim}
## # A tibble: 3 x 2
##   species       n
##   <fct>     <int>
## 1 Adelie      152
## 2 Chinstrap    68
## 3 Gentoo      124
\end{verbatim}

\hypertarget{using-geom_col}{%
\subsection{Using geom\_col()}\label{using-geom_col}}

For \texttt{geom\_col()} you need to have your data pre-counted. Let's
create a new data frame with counts per \texttt{species} in the
\texttt{penguins} dataset first.

\begin{Shaded}
\begin{Highlighting}[]
\CommentTok{\# create new data frame with species counts}
\NormalTok{species\_count\_data }\OtherTok{\textless{}{-}}\NormalTok{ penguins }\SpecialCharTok{\%\textgreater{}\%}
  \FunctionTok{count}\NormalTok{(species)}

\CommentTok{\# inspect new data frame}
\NormalTok{species\_count\_data}
\end{Highlighting}
\end{Shaded}

\begin{verbatim}
## # A tibble: 3 x 2
##   species       n
##   <fct>     <int>
## 1 Adelie      152
## 2 Chinstrap    68
## 3 Gentoo      124
\end{verbatim}

We now have a small data frame with two columns: \texttt{species} and
\texttt{n} -- a categorical and a numeric variables. We can now use
\texttt{geom\_col()} by first mapping one of the axes to the categorical
variable (i.e., \texttt{species}) and the other to the numeric variable
(i.e., \texttt{n}).

\begin{Shaded}
\begin{Highlighting}[]
\NormalTok{species\_count\_data }\SpecialCharTok{\%\textgreater{}\%}
  \FunctionTok{ggplot}\NormalTok{(}\FunctionTok{aes}\NormalTok{(}\AttributeTok{x =}\NormalTok{ species, }\AttributeTok{y =}\NormalTok{ n)) }\SpecialCharTok{+}
  \FunctionTok{geom\_col}\NormalTok{()}
\end{Highlighting}
\end{Shaded}

\begin{figure}

{\centering \includegraphics{scatterplots-barplots_code-along_files/figure-latex/geom_col_plot-1} 

}

\caption{Bar plot of penguin count mapping both axis.}\label{fig:geom_col_plot}
\end{figure}

Most of the time we will summarize the data in some way first, and then
use \texttt{geom\_col()} to create barplots, since having both the
categorical and the numeric variable explicitly represented in our
dataset allows for reordering the bars in an easier more straightforward
way.

\begin{Shaded}
\begin{Highlighting}[]
\NormalTok{species\_count\_data }\SpecialCharTok{\%\textgreater{}\%}
  \FunctionTok{ggplot}\NormalTok{(}\FunctionTok{aes}\NormalTok{(}\AttributeTok{x =} \FunctionTok{reorder}\NormalTok{(species, n), }
             \AttributeTok{y =}\NormalTok{ n)) }\SpecialCharTok{+}
  \FunctionTok{geom\_col}\NormalTok{() }\SpecialCharTok{+}
  \FunctionTok{labs}\NormalTok{(}\AttributeTok{x =} \StringTok{""}\NormalTok{)}
\end{Highlighting}
\end{Shaded}

\begin{figure}

{\centering \includegraphics{scatterplots-barplots_code-along_files/figure-latex/reordered_plot-1} 

}

\caption{Bar plot of penguin count mapping both axis with geom_col with species reordered by count.}\label{fig:reordered_plot}
\end{figure}

\hypertarget{dealing-with-ordered-categorical-variables}{%
\subsection{Dealing with ordered categorical
variables}\label{dealing-with-ordered-categorical-variables}}

Data from
\href{https://github.com/rfordatascience/tidytuesday/blob/master/data/2021/2021-12-21/readme.md}{tidytuesday}

\begin{Shaded}
\begin{Highlighting}[]
\NormalTok{starbucks }\OtherTok{\textless{}{-}} \FunctionTok{read\_csv}\NormalTok{(}\StringTok{\textquotesingle{}https://raw.githubusercontent.com/rfordatascience/tidytuesday/master/data/2021/2021{-}12{-}21/starbucks.csv\textquotesingle{}}\NormalTok{)}
\end{Highlighting}
\end{Shaded}

\begin{Shaded}
\begin{Highlighting}[]
\NormalTok{starbucks }\SpecialCharTok{\%\textgreater{}\%}
  \FunctionTok{group\_by}\NormalTok{(size) }\SpecialCharTok{\%\textgreater{}\%}
  \FunctionTok{summarize}\NormalTok{(}\AttributeTok{mean\_calories =} \FunctionTok{mean}\NormalTok{(calories),}
            \AttributeTok{n =} \FunctionTok{n}\NormalTok{()) }\SpecialCharTok{\%\textgreater{}\%}
  \FunctionTok{slice\_max}\NormalTok{(}\AttributeTok{order\_by =}\NormalTok{ n, }\AttributeTok{n =} \DecValTok{4}\NormalTok{) }\SpecialCharTok{\%\textgreater{}\%}
  \FunctionTok{mutate}\NormalTok{(}\AttributeTok{size =} \FunctionTok{factor}\NormalTok{(size,}
                       \AttributeTok{levels =} \FunctionTok{c}\NormalTok{(}\StringTok{"short"}\NormalTok{,}
                                  \StringTok{"tall"}\NormalTok{,}
                                  \StringTok{"grande"}\NormalTok{,}
                                  \StringTok{"venti"}\NormalTok{))) }\SpecialCharTok{\%\textgreater{}\%}
  \FunctionTok{ggplot}\NormalTok{(}\FunctionTok{aes}\NormalTok{(}\AttributeTok{x =}\NormalTok{ size,}
             \AttributeTok{y =}\NormalTok{ mean\_calories)) }\SpecialCharTok{+}
  \FunctionTok{geom\_col}\NormalTok{()}
\end{Highlighting}
\end{Shaded}

\begin{figure}

{\centering \includegraphics{scatterplots-barplots_code-along_files/figure-latex/starbucks_drinks-1} 

}

\caption{Mean calories across different starbucks drink sizes.}\label{fig:starbucks_drinks}
\end{figure}

\end{document}
