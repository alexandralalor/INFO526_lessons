% Options for packages loaded elsewhere
\PassOptionsToPackage{unicode}{hyperref}
\PassOptionsToPackage{hyphens}{url}
%
\documentclass[
]{article}
\usepackage{amsmath,amssymb}
\usepackage{lmodern}
\usepackage{iftex}
\ifPDFTeX
  \usepackage[T1]{fontenc}
  \usepackage[utf8]{inputenc}
  \usepackage{textcomp} % provide euro and other symbols
\else % if luatex or xetex
  \usepackage{unicode-math}
  \defaultfontfeatures{Scale=MatchLowercase}
  \defaultfontfeatures[\rmfamily]{Ligatures=TeX,Scale=1}
\fi
% Use upquote if available, for straight quotes in verbatim environments
\IfFileExists{upquote.sty}{\usepackage{upquote}}{}
\IfFileExists{microtype.sty}{% use microtype if available
  \usepackage[]{microtype}
  \UseMicrotypeSet[protrusion]{basicmath} % disable protrusion for tt fonts
}{}
\makeatletter
\@ifundefined{KOMAClassName}{% if non-KOMA class
  \IfFileExists{parskip.sty}{%
    \usepackage{parskip}
  }{% else
    \setlength{\parindent}{0pt}
    \setlength{\parskip}{6pt plus 2pt minus 1pt}}
}{% if KOMA class
  \KOMAoptions{parskip=half}}
\makeatother
\usepackage{xcolor}
\IfFileExists{xurl.sty}{\usepackage{xurl}}{} % add URL line breaks if available
\IfFileExists{bookmark.sty}{\usepackage{bookmark}}{\usepackage{hyperref}}
\hypersetup{
  pdftitle={Data Viz Critique 2},
  pdfauthor={Alexandra Lalor},
  hidelinks,
  pdfcreator={LaTeX via pandoc}}
\urlstyle{same} % disable monospaced font for URLs
\usepackage[margin=1in]{geometry}
\usepackage{color}
\usepackage{fancyvrb}
\newcommand{\VerbBar}{|}
\newcommand{\VERB}{\Verb[commandchars=\\\{\}]}
\DefineVerbatimEnvironment{Highlighting}{Verbatim}{commandchars=\\\{\}}
% Add ',fontsize=\small' for more characters per line
\usepackage{framed}
\definecolor{shadecolor}{RGB}{248,248,248}
\newenvironment{Shaded}{\begin{snugshade}}{\end{snugshade}}
\newcommand{\AlertTok}[1]{\textcolor[rgb]{0.94,0.16,0.16}{#1}}
\newcommand{\AnnotationTok}[1]{\textcolor[rgb]{0.56,0.35,0.01}{\textbf{\textit{#1}}}}
\newcommand{\AttributeTok}[1]{\textcolor[rgb]{0.77,0.63,0.00}{#1}}
\newcommand{\BaseNTok}[1]{\textcolor[rgb]{0.00,0.00,0.81}{#1}}
\newcommand{\BuiltInTok}[1]{#1}
\newcommand{\CharTok}[1]{\textcolor[rgb]{0.31,0.60,0.02}{#1}}
\newcommand{\CommentTok}[1]{\textcolor[rgb]{0.56,0.35,0.01}{\textit{#1}}}
\newcommand{\CommentVarTok}[1]{\textcolor[rgb]{0.56,0.35,0.01}{\textbf{\textit{#1}}}}
\newcommand{\ConstantTok}[1]{\textcolor[rgb]{0.00,0.00,0.00}{#1}}
\newcommand{\ControlFlowTok}[1]{\textcolor[rgb]{0.13,0.29,0.53}{\textbf{#1}}}
\newcommand{\DataTypeTok}[1]{\textcolor[rgb]{0.13,0.29,0.53}{#1}}
\newcommand{\DecValTok}[1]{\textcolor[rgb]{0.00,0.00,0.81}{#1}}
\newcommand{\DocumentationTok}[1]{\textcolor[rgb]{0.56,0.35,0.01}{\textbf{\textit{#1}}}}
\newcommand{\ErrorTok}[1]{\textcolor[rgb]{0.64,0.00,0.00}{\textbf{#1}}}
\newcommand{\ExtensionTok}[1]{#1}
\newcommand{\FloatTok}[1]{\textcolor[rgb]{0.00,0.00,0.81}{#1}}
\newcommand{\FunctionTok}[1]{\textcolor[rgb]{0.00,0.00,0.00}{#1}}
\newcommand{\ImportTok}[1]{#1}
\newcommand{\InformationTok}[1]{\textcolor[rgb]{0.56,0.35,0.01}{\textbf{\textit{#1}}}}
\newcommand{\KeywordTok}[1]{\textcolor[rgb]{0.13,0.29,0.53}{\textbf{#1}}}
\newcommand{\NormalTok}[1]{#1}
\newcommand{\OperatorTok}[1]{\textcolor[rgb]{0.81,0.36,0.00}{\textbf{#1}}}
\newcommand{\OtherTok}[1]{\textcolor[rgb]{0.56,0.35,0.01}{#1}}
\newcommand{\PreprocessorTok}[1]{\textcolor[rgb]{0.56,0.35,0.01}{\textit{#1}}}
\newcommand{\RegionMarkerTok}[1]{#1}
\newcommand{\SpecialCharTok}[1]{\textcolor[rgb]{0.00,0.00,0.00}{#1}}
\newcommand{\SpecialStringTok}[1]{\textcolor[rgb]{0.31,0.60,0.02}{#1}}
\newcommand{\StringTok}[1]{\textcolor[rgb]{0.31,0.60,0.02}{#1}}
\newcommand{\VariableTok}[1]{\textcolor[rgb]{0.00,0.00,0.00}{#1}}
\newcommand{\VerbatimStringTok}[1]{\textcolor[rgb]{0.31,0.60,0.02}{#1}}
\newcommand{\WarningTok}[1]{\textcolor[rgb]{0.56,0.35,0.01}{\textbf{\textit{#1}}}}
\usepackage{graphicx}
\makeatletter
\def\maxwidth{\ifdim\Gin@nat@width>\linewidth\linewidth\else\Gin@nat@width\fi}
\def\maxheight{\ifdim\Gin@nat@height>\textheight\textheight\else\Gin@nat@height\fi}
\makeatother
% Scale images if necessary, so that they will not overflow the page
% margins by default, and it is still possible to overwrite the defaults
% using explicit options in \includegraphics[width, height, ...]{}
\setkeys{Gin}{width=\maxwidth,height=\maxheight,keepaspectratio}
% Set default figure placement to htbp
\makeatletter
\def\fps@figure{htbp}
\makeatother
\setlength{\emergencystretch}{3em} % prevent overfull lines
\providecommand{\tightlist}{%
  \setlength{\itemsep}{0pt}\setlength{\parskip}{0pt}}
\setcounter{secnumdepth}{-\maxdimen} % remove section numbering
\ifLuaTeX
  \usepackage{selnolig}  % disable illegal ligatures
\fi

\title{Data Viz Critique 2}
\author{Alexandra Lalor}
\date{2022-05-02}

\begin{document}
\maketitle

\hypertarget{background}{%
\subsection{Background}\label{background}}

For this data viz critique, I will be updating figure 2 from the article
\href{https://journals.plos.org/plosbiology/article?id=10.1371/journal.pbio.3000763}{Low
Availability of Code in Ecology: A Call for Urgent Action (Culina et
al.~2020)}, using code
\href{https://zenodo.org/record/3833928\#.Ym9XNNrMI2w}{004\_plotting.R}
from the original ariticle. I referenced data collection
\href{https://asanchez-tojar.github.io/code_in_ecology/supporting_information.html}{methods}
as supporting information and used
\href{https://rpkgs.datanovia.com/ggpubr/reference/ggdotchart.html}{Cleveland's
Dot Plots} as the primary figure.

\hypertarget{methods}{%
\subsection{Methods}\label{methods}}

This paper is a review of ecology journals, and the accessibility of
code in articles published by these journals. The authors randomly
sampled articles published in ecology journals for which code-sharing
has been either mandatory or encouraged since June 2015 at the latest.

A total of 14 out of 96 (15\%) ecological journals encourage
code-sharing policies. From these journals, a random sample of 400
articles was taken. Articles were further screened for relevancy,
leaving 346 studies in this review.

Based on the information collected, the authors scored the journals'
code-sharing policies as:

\begin{itemize}
\tightlist
\item
  \textbf{encouraged}: publication of the code is explicitly encouraged,
  but not required
\item
  \textbf{mandatory}: code must be published together with the article
\item
  \textbf{encouraged/mandatory}: when the wording made it difficult to
  judge if code publication is encouraged or required
\end{itemize}

\hypertarget{read-in-original-data}{%
\subsubsection{Read in original data}\label{read-in-original-data}}

\begin{Shaded}
\begin{Highlighting}[]
\CommentTok{\#load packages}
\FunctionTok{library}\NormalTok{(ggpubr)}
\end{Highlighting}
\end{Shaded}

\begin{verbatim}
## Loading required package: ggplot2
\end{verbatim}

\begin{Shaded}
\begin{Highlighting}[]
\FunctionTok{library}\NormalTok{(tidyverse)}
\end{Highlighting}
\end{Shaded}

\begin{verbatim}
## -- Attaching packages --------------------------------------- tidyverse 1.3.1 --
\end{verbatim}

\begin{verbatim}
## v tibble  3.1.6     v dplyr   1.0.8
## v tidyr   1.2.0     v stringr 1.4.0
## v readr   2.1.2     v forcats 0.5.1
## v purrr   0.3.4
\end{verbatim}

\begin{verbatim}
## -- Conflicts ------------------------------------------ tidyverse_conflicts() --
## x dplyr::filter() masks stats::filter()
## x dplyr::lag()    masks stats::lag()
\end{verbatim}

\begin{Shaded}
\begin{Highlighting}[]
\CommentTok{\#read in data}
\NormalTok{full.journal.info }\OtherTok{\textless{}{-}} \FunctionTok{read\_csv}\NormalTok{(}\StringTok{"data\_raw/viz\_critique/code\_availability.csv"}\NormalTok{)}
\end{Highlighting}
\end{Shaded}

\begin{verbatim}
## New names:
## * `` -> ...1
\end{verbatim}

\begin{verbatim}
## Rows: 14 Columns: 7
## -- Column specification --------------------------------------------------------
## Delimiter: ","
## chr (3): Journal, Policy, abbreviations
## dbl (4): ...1, codepublished, total, percentage
## 
## i Use `spec()` to retrieve the full column specification for this data.
## i Specify the column types or set `show_col_types = FALSE` to quiet this message.
\end{verbatim}

\hypertarget{original-figure}{%
\subsubsection{Original figure}\label{original-figure}}

\begin{Shaded}
\begin{Highlighting}[]
\FunctionTok{ggdotchart}\NormalTok{(full.journal.info, }\AttributeTok{x =} \StringTok{"abbreviations"}\NormalTok{, }\AttributeTok{y =} \StringTok{"percentage"}\NormalTok{,}
                      \AttributeTok{color =} \StringTok{"Policy"}\NormalTok{,}
                      \AttributeTok{palette =} \FunctionTok{c}\NormalTok{(}\StringTok{"\#00AFBB"}\NormalTok{, }\StringTok{"\#E7B800"}\NormalTok{, }\StringTok{"\#FC4E07"}\NormalTok{),}
                      \AttributeTok{sorting =} \StringTok{"descending"}\NormalTok{,}
                      \AttributeTok{add =} \StringTok{"segments"}\NormalTok{,}
                      \AttributeTok{rotate =} \ConstantTok{TRUE}\NormalTok{,}
                      \AttributeTok{group =} \StringTok{"Policy"}\NormalTok{,}
                      \AttributeTok{xlab =} \StringTok{""}\NormalTok{,}
                      \AttributeTok{ylab =} \StringTok{"Percentage (\%) of articles publishing some code"}\NormalTok{,}
                      \AttributeTok{dot.size =} \DecValTok{8}\NormalTok{,}
                      \AttributeTok{label =} \FunctionTok{paste0}\NormalTok{(}\FunctionTok{round}\NormalTok{(full.journal.info}\SpecialCharTok{$}\NormalTok{codepublished,}\DecValTok{0}\NormalTok{),}\StringTok{"/"}\NormalTok{,}
                                     \FunctionTok{round}\NormalTok{(full.journal.info}\SpecialCharTok{$}\NormalTok{total,}\DecValTok{0}\NormalTok{)),}
                      \AttributeTok{font.label =} \FunctionTok{list}\NormalTok{(}\AttributeTok{color =} \StringTok{"black"}\NormalTok{, }\AttributeTok{size =} \DecValTok{7}\NormalTok{,}\AttributeTok{vjust =} \FloatTok{0.5}\NormalTok{),}
                      \AttributeTok{ggtheme =} \FunctionTok{theme\_pubr}\NormalTok{())}
\end{Highlighting}
\end{Shaded}

\includegraphics{viz_critique_2_files/figure-latex/unnamed-chunk-2-1.pdf}

\hypertarget{orignial-figure-comments}{%
\subsubsection{Orignial figure
comments}\label{orignial-figure-comments}}

\begin{enumerate}
\def\labelenumi{\arabic{enumi}.}
\item
  The category ``Encouraged/Mandatory'' is confusing, because I don't
  understand the overlap between these two categories. Looking back at
  the methods, we find that the definition of
  \textbf{``encouraged/mandatory''} is when the wording made it
  \emph{difficult to judge if code publication is encouraged or
  required}. Based off this definition, I would argue that if code
  sharing is not explicitly required then it is not mandatory.
  Therefore, I will redefine these categories when creating my new
  visualization.
\item
  The fractions within the colored circles are difficult to read, and
  only upon close examination do we see the wide variation in number of
  articles reviewed per journal. I'd like to create a visualization
  which shows percentage as well as total number of articles under
  review.
\item
  I would add a title to this figure.
\item
  I like the dotplot display of data overall, so I want to maintain
  this. I would change the color of the ``Mandatory'' policy to
  something besides red, because code sharing should be viewed as a
  positive thing while red conveys negativity.
\item
  What is the main focus of this figure? I think it's to show me that
  percentage of articles publishing code is very similar regardless if
  policies require code publishing. My updated graph will try to
  communicate the same message, but more clearly. I can do this by
  adjusting the policy categorizations and adding a title.
\end{enumerate}

\hypertarget{mutate-data}{%
\subsubsection{Mutate data}\label{mutate-data}}

\begin{Shaded}
\begin{Highlighting}[]
\CommentTok{\#mutate policy column, add count column, merge}
\NormalTok{code\_availability\_1 }\OtherTok{\textless{}{-}}\NormalTok{ full.journal.info }\SpecialCharTok{\%\textgreater{}\%} 
  \FunctionTok{mutate}\NormalTok{(}\AttributeTok{Policy\_sort =} \FunctionTok{ifelse}\NormalTok{(Policy }\SpecialCharTok{==} \StringTok{"Encouraged/Mandatory"}\NormalTok{, }\StringTok{"Encouraged\_total"}\NormalTok{, }
                              \FunctionTok{ifelse}\NormalTok{(Policy }\SpecialCharTok{==} \StringTok{"Encouraged"}\NormalTok{, }\StringTok{"Encouraged\_total"}\NormalTok{, }\StringTok{"Mandatory\_total"}\NormalTok{))) }\SpecialCharTok{\%\textgreater{}\%} 
  \FunctionTok{mutate}\NormalTok{(}\AttributeTok{count =}\NormalTok{ total) }\SpecialCharTok{\%\textgreater{}\%} 
  \FunctionTok{mutate}\NormalTok{(}\AttributeTok{Policy =} \StringTok{"Total"}\NormalTok{) }\SpecialCharTok{\%\textgreater{}\%} 
  \FunctionTok{mutate}\NormalTok{(}\AttributeTok{percentage =} \FunctionTok{round}\NormalTok{((count}\SpecialCharTok{/}\NormalTok{total)}\SpecialCharTok{*}\DecValTok{100}\NormalTok{))}

\NormalTok{code\_availability\_2 }\OtherTok{\textless{}{-}}\NormalTok{ full.journal.info }\SpecialCharTok{\%\textgreater{}\%} 
  \FunctionTok{mutate}\NormalTok{(}\AttributeTok{Policy\_sort =} \FunctionTok{ifelse}\NormalTok{(Policy }\SpecialCharTok{==} \StringTok{"Encouraged/Mandatory"}\NormalTok{, }\StringTok{"Encouraged\_published"}\NormalTok{, }
                         \FunctionTok{ifelse}\NormalTok{(Policy }\SpecialCharTok{==} \StringTok{"Encouraged"}\NormalTok{, }\StringTok{"Encouraged\_published"}\NormalTok{, }\StringTok{"Mandatory\_published"}\NormalTok{))) }\SpecialCharTok{\%\textgreater{}\%} 
  \FunctionTok{mutate}\NormalTok{(}\AttributeTok{count =}\NormalTok{ codepublished) }\SpecialCharTok{\%\textgreater{}\%} 
  \FunctionTok{mutate}\NormalTok{(}\AttributeTok{Policy =} \FunctionTok{ifelse}\NormalTok{(Policy\_sort }\SpecialCharTok{==} \StringTok{"Encouraged\_published"}\NormalTok{, }\StringTok{"Encouraged"}\NormalTok{, }\StringTok{"Mandatory"}\NormalTok{)) }\SpecialCharTok{\%\textgreater{}\%} 
  \FunctionTok{mutate}\NormalTok{(}\AttributeTok{percentage =} \FunctionTok{round}\NormalTok{((count}\SpecialCharTok{/}\NormalTok{total)}\SpecialCharTok{*}\DecValTok{100}\NormalTok{))}

\NormalTok{code\_availability }\OtherTok{\textless{}{-}} \FunctionTok{rbind}\NormalTok{(code\_availability\_1,code\_availability\_2)}
\end{Highlighting}
\end{Shaded}

\hypertarget{updated-figure}{%
\subsubsection{Updated figure}\label{updated-figure}}

\begin{Shaded}
\begin{Highlighting}[]
\CommentTok{\#visualize}
\NormalTok{code\_availability }\SpecialCharTok{\%\textgreater{}\%}
  \FunctionTok{ggdotchart}\NormalTok{(}\AttributeTok{x =} \StringTok{"abbreviations"}\NormalTok{, }
             \AttributeTok{y =} \StringTok{"count"}\NormalTok{,}
             \AttributeTok{color =} \StringTok{"Policy"}\NormalTok{,}
             \AttributeTok{palette =} \FunctionTok{c}\NormalTok{(}\StringTok{"\#00AFBB"}\NormalTok{, }\StringTok{"\#E7B800"}\NormalTok{, }\StringTok{"\#D3D3D3"}\NormalTok{),}
             \AttributeTok{sorting =} \StringTok{"descending"}\NormalTok{,}
             \AttributeTok{rotate =} \ConstantTok{TRUE}\NormalTok{,}
             \AttributeTok{group =} \StringTok{"Policy"}\NormalTok{,}
             \AttributeTok{add =} \StringTok{"segments"}\NormalTok{,}
             \AttributeTok{title =} \StringTok{"Ecology journals with code{-}sharing policies }\SpecialCharTok{\textbackslash{}n}\StringTok{still have low availability of code"}\NormalTok{,}
             \AttributeTok{xlab =} \StringTok{""}\NormalTok{,}
             \AttributeTok{ylab =} \StringTok{"Number of articles with published code (color) of those reviewed (grey)"}\NormalTok{,}
             \AttributeTok{dot.size =} \DecValTok{7}\NormalTok{,}
             \AttributeTok{label =} \FunctionTok{paste0}\NormalTok{(code\_availability}\SpecialCharTok{$}\NormalTok{percentage,}\StringTok{"\%"}\NormalTok{),}
             \AttributeTok{label.select =} \FunctionTok{list}\NormalTok{(}\AttributeTok{criteria =} \StringTok{"Policy \%in\% c(\textquotesingle{}Encouraged\textquotesingle{},\textquotesingle{}Mandatory\textquotesingle{})"}\NormalTok{),}
             \AttributeTok{font.label =} \FunctionTok{list}\NormalTok{(}\AttributeTok{color =} \StringTok{"black"}\NormalTok{, }\AttributeTok{size =} \DecValTok{7}\NormalTok{,}\AttributeTok{vjust =} \FloatTok{0.5}\NormalTok{),}
             \AttributeTok{ggtheme =} \FunctionTok{theme\_pubr}\NormalTok{())}
\end{Highlighting}
\end{Shaded}

\includegraphics{viz_critique_2_files/figure-latex/unnamed-chunk-4-1.pdf}

\hypertarget{updated-figure-comments}{%
\subsubsection{Updated figure comments}\label{updated-figure-comments}}

In this figure, number of articles is shown on the x-axis (rather than
percent). Grey circles show the totoal number of articles under review,
and colored circles show the number of articles with published code.
Percentages are added to colored circles to show how each colored circle
compares to the grey circles.

By updating the figure, I can more clearly see the difference between
``encouraged'' and ``mandatory'' policies, and it becomes evident that
mandatory policies are not successful at improving code access. The
figure now has a clear title which communicates this message, and more
nuanced display of number of total articles, number of articles with
published code, and percent articles as an additional aid to show low
code availability.

\end{document}
